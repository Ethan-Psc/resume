\documentclass{resume}

\newcommand{\en}[1]{#1}
\newcommand{\zh}[1]{}

\zh{\usepackage{xeCJK}}

\begin{document}

\name{\en{Yongjia Tan}\zh{谭永佳}}
\basicInfo{
      \email{897795208@qq.com} \textperiodcentered\
      \github[MrEphemera]{https://github.com/MrEphemera} \textperiodcentered\
}

\section{\en{Education}\zh{教育经历}}
\en{\datedsubsection{\textbf{South China University of Technology}, Bachelor}{2019 -- 2023 (Expected)}}
\zh{\datedsubsection{\textbf{华南理工大学}, 本科}{2019/09 -- 至今}}
\begin{itemize}
      \item \en{Major: Computer Engineering, \textit{Computer Science and Technology}}
            \zh{计算机科学与工程 计算机科学与技术}
\end{itemize}

\section{\en{Skills}\zh{技能}}
\begin{itemize}[parsep=0.25ex]
      \item \en{\textbf{Programming Languages}:
                  experienced in HTML/CSS/JavaScript,comfortable with TypeScript/C++/Java (in order).}
            \zh{\textbf{编程语言}:
                  熟悉 HTML/CSS/JavaScript,
                  了解 TypeScript/C++/Java。}

      \item \en{\textbf{System}:
                  familiar with computer network concepts such as HTTP/HTTPS, TCP/IP, DNS and so on
                  comfortable with operation system concepts
                 }
            \zh{\textbf{系统}:
                  熟悉HTTP/HTTPS,TCP/IP,DNS等网络协议,了解操作系统的相关概念 }

      \item \en{\textbf{Front-end frame}:
                  have experience in Vue.js,comfortable with React.js}
            \zh{\textbf{前端框架}:
                  熟悉Vue.js,了解React.js }

      \item \en{\textbf{Back-end}:
                  comfortable with Node.js, Koa}
            \zh{\textbf{后端}:
                  了解Node.js,Koa}

      \item \en{\textbf{Developing Tools}:
                  experienced in Linux-based programming,
                  have experience with team tools like Git, etc.}
            \zh{\textbf{开发工具}:
                  熟悉 Linux,有Git 等团队合作工具的经验}
\end{itemize}

\section{\en{Work Experience}\zh{工作经历}}
\en{\datedsubsection{\textbf{\href{https://www.bytedance.com/}{Baidu}}, Beijing, China}{05/2022 -- 09/2022}}
\zh{\datedsubsection{\textbf{\href{https://www.bytedance.com/}{百度}}}{2022/05 -- 2022/09}}
\en{\role{Baidu search products}{Front-end web Engineer Intern}}
\zh{\role{百度搜索产品}{Rust 研发实习}}
\begin{itemize}
      \item \en{Responsible for the construction and maintenance of the E2E low-code verification capability of the department, and verify fields in the form of configuration items}
            \zh{负责本部门E2E低代码校验能力的构建与维护,采用配置项的形式校验字段}
      \item \en{Responsible for maintaining and upgrading the E2E management platform}
            \zh{负责维护升级E2E相关管理平台}
      \item \en{Designed a multi-process current-limiting architecture for asynchronous execution of E2E tests, and the running speed is improved by remote control of the real machine}
            \zh{设计了多进程限流架构用于异步执行E2E测试,通过远程调控真机提升运行速度}
\end{itemize}

\section{\en{Portfolios}\zh{个人项目}}
\begin{itemize}[parsep=0.25ex]
      \item \textbf{\href{https://github.com/MrEphemera/vue-music}{vue-music}}:
            \en{The music player WebApp developed based on the Vue3.
                Implemented the recommendation page, the singer page and the core player component
            }
            \zh{基于Vue3全家桶开发的音乐播放器WebApp,
                实现了推荐页面,歌手页面以及内核的播放器组件}
       \item 
            \en{Developed with Vue3, citing third-party plug-ins such as better-scroll and good-storage to assist development}
            \zh{使用上线的后端数据接口,采用devServer对数据进行转换与处理  }
       \item 
            \en{Use the online back-end data interface, and use devServer to convert and process data}
            \zh{采用Vue3开发,引用better-scroll,good-storage等第三方插件辅助开发 }
       \item 
            \en{Use Vuex to cooperate with local storage to manage the global state and record the currently playing song}
            \zh{采用Vuex配合本地存储管理全局状态,记录当前播放的歌曲}
      \item 
            \en{Using the composition API modular idea, the player is divided into lyrics, cd record rotation, finger interaction and other related logic}
            \zh{采用CompositionAPI模块化思想,将播放器分为歌词、cd 唱片旋转、手指交互等相关逻辑 }
      \item 
            \en{Use Vue Router to configure routing view for page switching and realize routing transition interactive animation}
            \zh{采用Vue Router配置路由视图进行页面切换和实现路由过渡交互动画  }

            
      \item \textbf{\href{https://github.com/MrEphemera/CreateUI}{CreateUI}}:
            \en{Development of CreateUI Component Library Based on Vue2.x。
                Implemented more than ten components and build the corresponding documentation website
            }
            \zh{基于Vue2.x的CreateUI组件库开发,
                实现10+组件,并搭建相应的文档网站}
      \item 
            \en{Developed with Vue2.x+Less to realize the effect display of component library}
            \zh{采用Vue2.x+Less开发,实现组件库效果展示 }
      \item 
            \en{Use routing to introduce lazy loading and image lazy loading to speed up homepage time}
            \zh{采用路由引入懒加载方式和图片懒加载,加快首页时间}
      \item 
            \en{Use the babel-plugin-component plug-in to import components on demand}
            \zh{采用babel-plugin-component插件实现组件按需引入}
      \item 
            \en{Use the Less framework to define CSS global variables and unify project colors}
            \zh{采用Less框架定义CSS全局变量,统一项目色彩}
    
\end{itemize}

\end{document}
