\documentclass{resume}

\newcommand{\en}[1]{#1}
\newcommand{\zh}[1]{}

\zh{\usepackage{xeCJK}}
\zh{\setCJKmainfont{思源宋体}}
\zh{\setCJKsansfont{思源黑体}}
\zh{\setCJKmonofont{思源黑体}}

\begin{document}

\name{\en{Peng Sha}\zh{沙 \ 鹏}}
\basicInfo{
  \email{jack@shapeng1998.com} \textperiodcentered\
  \github[shapeng1998]{https://github.com/shapeng1998} \textperiodcentered\
  \homepage[Blog]{https://blog.shapeng1998.com}
}

\section{\en{Education}\zh{教育经历}}
\en{\datedsubsection{\textbf{Southeast University}, Postgraduate}{09/2020 -- Present}}
\zh{\datedsubsection{\textbf{东南大学}, 在读硕士研究生}{2020/09 -- 至今}}
\begin{itemize}
  \item \en{Major: Control science and Engineering, School of Automation. Anticipated graduation date: 06/2023}
        \zh{控制科学与工程,自动化学院,2023 年毕业}
\end{itemize}

\en{\datedsubsection{\textbf{Beijing University of Chemical Technology}, Bachelor's Degree}{09/2016 -- 06/2020}}
\zh{\datedsubsection{\textbf{北京化工大学}, 本科}{2016/09 -- 2020/06}}
\begin{itemize}
  \item \en{Major: Automation, Excellent engineer class, School of information science and technology.}
        \zh{自动化,卓越工程师实验班,信息科学与技术学院}
  \item \en{GPA: 3.67/4.33 (Ranked 7 out of 199) Exempted to Southeast University for master's degree in engineering}
        \zh{GPA: 3.67/4.33 (排名 7/199) 免试保送至东南大学攻读工学硕士学位}
\end{itemize}

\section{\en{Work Experience}\zh{工作经历}}
\en{\datedsubsection{\textbf{\href{https://www.bytedance.com/en/}{ByteDance Inc.}}, Shanghai, China}{07/2021 -- 09/2021}}
\zh{\datedsubsection{\textbf{\href{https://www.bytedance.com/zh/}{字节跳动(ByteDance Inc.)}}}{2021/07 -- 2021/09}}
\en{\role{Lark Video Conference Room}{R\&D Intern, React.js/Typescript}}
\zh{\role{飞书会议室解决方案团队}{前端研发实习}}
\begin{itemize}
  \item \en{Responsible for Lark Video Conference Admin Platform R\&D work related to vertical business, using React.js, Mobx and other tools for development and maintenance.}
        \zh{负责飞书会议室 Admin 中会议室模块和视频会议模块相关垂直业务的前端研发,使用 React.js 和 Mobx 等工具进行业务等开发和维护}
  \item \en{Participated in R\&D work of key accounts, involving multi-terminal cooperative development and multi-terminal joint debugging, including joint debugging of front and back ends and desktop ends.}
        \zh{参与重点客户相关需求的开发,涉及到了多端合作开发与多端联合调试,包括前后端以及桌面端的联合调试,学到了很多联合调试开发经验}
  \item \en{Learned a lot about the practical experience of enterprise development and refactor some business code.}
        \zh{学到了很多关于 React.js,Typescript 以及企业级开发流程的实战经验,并对部分业务代码进行重构,例如使用 React Hooks 和 functional components 替代 class based components}
  \item \en{\textbf{Technology Stack}: React.js, Typescript, Mobx.}
        \zh{\textbf{技术栈}: React.js, Typescript, Mobx}
\end{itemize}

\section{\en{Portfolios}\zh{个人项目}}
\en{\datedsubsection{\textbf{ByteDance Front-end Training Camp}}{11/2020 -- 12/2020}}
\zh{\datedsubsection{\textbf{字节跳动前端训练营}}{2020/11 -- 2020/12}}
\en{2048 game for multi users}
\zh{2048 多人在线小游戏}
\begin{itemize}
  \item \en{As the team leader of the project, responsible for the back-end development and deployment.}
        \zh{作为项目的组长, 负责项目的后端开发以及前后端整合上线}
  \item \en{Using Express, Socket.IO and MongoDB to develop the back-end service.}
        \zh{使用 Express, Socket.IO 以及 MongoDB 进行后端开发}
  \item \en{Using Github Actions and Docker to implement the CI/CD pipeline.}
        \zh{使用 Github Actions 和 Docker 实现前后端的 CI/CD}
  \item \en{Learn React.js during the project and master the basic usage.}
        \zh{在项目中自学 React.js 并掌握基本使用}
  \item \en{\textbf{Technology Stack}: React.js, Express, Typescript, Socket.IO, Docker.}
        \zh{\textbf{技术栈}: React.js, Express, Typescript, Socket.IO, Docker}
\end{itemize}
\datedsubsection{\textbf{notion-blog}}{https://github.com/shapeng1998/notion-blog}
\en{A notion based blog template}
\zh{基于 Notion 的个人博客模板}
\begin{itemize}
  \item \en{Using Notion as a blog CMS to write articles, support dark mode and have friendly SEO.}
        \zh{使用 Notion 作为 CMS 发表博客,支持黑暗模式和良好的 SEO}
  \item \en{Build the blog based on SSG and ISR features of Next.js with 95-100\% Lighthouse scores。}
        \zh{基于 Next.js 的 SSG 和 ISR 特性构建博客,LightHouse 评分高达 95-100\%}
  \item \en{Customize the blog's style with Tailwind CSS.}
        \zh{使用 Tailwind CSS 定制化博客的样式}
  \item \en{Using react-notion-x to render the blog content and Prism to highlight the code blocks.}
        \zh{使用 react-notion-x 渲染博客内容,使用 Prism 对代码块进行高亮}
  \item \en{\textbf{Technology Stack}: Next.js, Tailwind CSS, Typescript.}
        \zh{\textbf{技术栈}: Next.js, Tailwind CSS, Typescript}
\end{itemize}

\section{\en{Research \& Academic Experience}\zh{研究经历}}
\en{\datedsubsection{\textbf{Institute of Automation, Chinese Academy of Sciences}, Beijing, China}{04/2019 -- 06/2020}}
\zh{\datedsubsection{\textbf{中国科学院自动化研究所}}{2019/04 -- 2020/06}}
\en{\role{State Key Laboratory of Complex System Management and Control}{Research Intern}}
\zh{\role{复杂系统管理与控制国家重点实验室}{科研实习}}
\begin{itemize}
  \item \en{Participate in the development of background management system for edge computing.}
        \zh{参与构建面向边缘计算的后台管理系统开发}
  \item \en{The front-end is based on Vue.js, and the back-end is developed based on Flask and PostgreSQL}
        \zh{前端基于 Vue.js 进行开发,后端基于 Flask 和 PostgreSQL 进行开发}
  \item \en{Building images for multi-architecture platforms using Docker Buildx.}
        \zh{使用 Docker Buildx 构建多系统架构支持的 Docker 镜像并部署}
\end{itemize}

\section{\en{Skills}\zh{技能}}
\begin{itemize}[parsep=0.25ex]
  \item \en{\textbf{Programming Language}: \textbf{multilingual} (not limited to any specific language), and experienced in Typescript, comfortable with Python/Java/C++}
        \zh{\textbf{编程语言}: \textbf{泛语言}(编程不受特定语言限制),熟悉 Typescript,了解 Python/Java/C++ 等}
  \item \en{\textbf{Developing Tool}: familiar with Linux-based programming, have experience with team tools like Git, etc.}
        \zh{\textbf{开发工具}: 熟悉 Linux,有 Git 等团队协作工具的使用经验}
\end{itemize}

\section{\en{Miscellaneous}\zh{杂项}}
\begin{itemize}[parsep=0.25ex]
  \item \en{\textbf{Selected Courses}: OS, Network, Algorithm.}
        \zh{\textbf{主修课程}: 操作系统、计算机网络、算法设计与分析}
  \item \en{\textbf{Language Level}: English CET-6: 618 points, able to conduct daily conversation and essay reading.}
        \zh{\textbf{语言水平}: 英语 CET-6: 618 分,能够进行日常对话和论文阅读}
  \item \en{\textbf{First Prize in Beijing}, National College Student Electronic Design Competition 2019}
        \zh{\textbf{北京市一等奖}, 2019 年全国大学生电子设计竞赛}
\end{itemize}

\end{document}
